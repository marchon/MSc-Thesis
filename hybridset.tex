\chapter{Generalized Partitions}
\section{Piecewise Functions}
\doublespacing

The perennial example of a piecewise function is $\mathrm{abs}:\mathbb{R} \to \mathbb{R}_{+} \cup \{ 0 \}$ given in the form:

\begin{equation}
\mathrm{abs}(x) = 
  \left\{
     \begin{array}{lr}
       -x & : x < 0 \\
       x & : x \geq 0
     \end{array}
   \right.
\end{equation}

If $x < 0$ then the first case is evaluated, otherwise, if $x \geq 0$ the second case is evaluated.

We can also think of $\mathbb{R}$ as partitioned by $\mathbb{R}_{+} \cup \{ 0 \} $ and $\mathbb{R}_{-}$ with a sub-function attached to each.

This easily generalizes for a function $f$ for which $\{ P_i \}$ partitions the domain:

\begin{equation}
f(x) = 
  \left\{
     \begin{array}{lr}
       f_1(x) & : x \in P_1 \\
       f_2(x) & : x \in P_2 \\ 
       \vdots & \vdots \\
       f_n(x) & : x \in P_n
     \end{array}
   \right.
\end{equation}

An alternate representation uses the restriction of function

\begin{definition}
We define $f^S$, the restriction of a function $f$ to a set $S$, by:

\begin{equation}
f^S(x) = 
  \left\{
     \begin{array}{lr}
       f(x) & : x \in S \\
       \bot & : x \notin S
     \end{array}
   \right.
\label{eq_fP}
\end{equation}

where $\bot$ denotes ``undefined''.
\end{definition}

\begin{definition}
Define $\fjoin$, the \emph{join} of two functions, $f$ and $g$ by:

\begin{equation}
f \fjoin g =  
  \left\{
     \begin{array}{lr}
       f(x) & \text{if } g(x) = \bot \\
       g(x) & \text{if } f(x) = \bot \\
       \bot & otherwise
     \end{array}
   \right.
\end{equation}
\end{definition}

which would allow us to re-write our previous definition of (\ref{eq_fP}) as:

\begin{equation}
f^P = f^{P_1} \fjoin f^{P_2} \fjoin ... \fjoin f^{P_n}
\end{equation}

But we must be careful as this definition is not associative.

Let $x \in A \cap B \cap C$, then $( (f^A \fjoin g^B ) \fjoin h^C )(x) = h(x)$ but $( f^A \fjoin ( g^B \fjoin h^C )(x) = f(x)$

Other conventions exist, for example \emph{Maple}'s 
\texttt{piecewise(cond\_1, f\_1, cond\_2, f\_2, ..., cond\_n, f\_n, f\_otherwise)} 
effectively uses a short-circuted $\fjoin$ 
and takes the first \texttt{f\_i} such that \texttt{cond\_i} evaluates to \emph{true}.

This approach simply trades associativity for commutivity.

Sign function really wants to be 2 ``partitions'' with overlap at 0. (3 pages total)

\newpage \addtocounter{page}{2}

\section{Hybrid Sets}

\begin{definition}
Hybrid Set, Set of all hybrid sets on a universe
\end{definition}

\begin{definition}
Multiplicity, Member, $\in^n$, $\in$, $\notin$, Support, $\emptyset$
\end{definition}

Notation

\begin{definition}
$\oplus , \ominus , \otimes$, scalar multiplication
\end{definition}

\begin{definition}
Disjointness
\end{definition}

\begin{definition}
Reducibility, $\mathcal{R}(H) = \mathrm{supp}(H)$
\end{definition}

(2 pages)

\begin{example}{Rational number arithmetic or $\mathcal{H}(\mathbb{P}) \sim \mathbb{Q}_+$ } (1 page)
\begin{equation}
 20/9 * 15/8 = \hset[5^1, 2^2, 3^{-2}] \oplus \hset[5^1, 3^1, 2^{-3}] = \hset[5^2, 2^{-1}, 3^{-1}] = 25/6
\end{equation}
\end{example}


%Monic polynomial
\begin{example} Roots and Asymptotes of rational functions (holes are missed).
\begin{equation}
\frac{(x-2)}{(x-1)^2(x+1)} = \{\!| 2^1, 1^{-2}, -1^{-1} |\!\}
\end{equation}
\end{example}



\newpage \addtocounter{page}{2}

\section{Hybrid Functions}
\begin{definition}
$f^A = ...$
\end{definition}

\begin{definition}
$\mathcal{R}(f^A) = ...$
\end{definition}

\begin{definition}
$f^F \hjoin[] g^G = ...$
\end{definition}

\begin{definition}
Compatibility
\end{definition}

\begin{definition}
$f^F \hjoin[*] g^G = ...$
\end{definition}

(5 pages)

\begin{example}{Piece-wise functions} (2 pages)
\begin{align*}
(f * g) (x) &= \hset[ f_1(x)^{A_1} , f_2(x)^{A_2}] * \hset[g_1(x)^{B_1} , g_2(x)^{B_2}] \\
 &= \hset[(f_1(x)*g_1(x))^{A_1}] \hjoin[*] \hset[(f_2(x)*g_1(x))^{B_1 \ominus A_2}] \hjoin[*] \hset[(f_2(x) * g_2(x))^{B_2}]
\end{align*}
\end{example}

\newpage \addtocounter{page}{6}

\section{Pseudo-functions}
\begin{definition}
Refinement
\end{definition}

\begin{example}
Refinement of intervals
\end{example}

\begin{definition}
Pseudo-function
\end{definition}

Properties of pseudo-functions

(2 pages)

\begin{example}
Repeat piece-wise function example with unsafe points (1 page)
\begin{equation}
(-x^2+2)^{[-1,1]} \hjoin[] \left( \frac{1}{x^2} \right)^{[-\infty, -1) \oplus (1, \infty]}
\end{equation}
\end{example}

\newpage \addtocounter{page}{2}