\chapter{Conclusions}
\doublespacing

The primary objective of this thesis was to extend \cite{carette2010} by investigating further applications of hybrid sets and functions. 
Although this paper was focused on results for integration and Petri net graphs, this is not to take away from the ``smaller'' results shown along the way.
As a first example, we showed that, (notational choice: $\mathbb{Z}^\mathbb{P}$, $h( \mathbb{P} )$, $\mathcal{H}(\mathbb{P})$?), the hybrid sets over prime numbers is equivalent to $\mathbb{Q}_+$.



Hybrid sets come into their own within the context of hybrid functions as we saw with arithmetic on piecewise functions and symbolic matrices.
Combined with tricks from linear algebra, the usage of hybrid functions allowed for large decreases in both cases.
Hybrid pseudo-functions leave a function associated with an element unevalutated and allow for algebra to be performed on the domains before requesting any functions be evaluated.



Hybrid functions were shown to be a good model for domains of integration.
An atlas can succintly be defined in terms of a set of hybrid relation over a universe of Euclidean rectangles mapping to a common manifold.
Principle of Inclusion-Exclusion was used instead of the typical \emph{partitions of unity} to reduce the atlas to it's support.
Unlike some of the previous examples, any leftover negative terms produced by PIE are completely well-founded and have natural geometric interpretations.
Moreover, $\partial$, the boundary operator on a $k$-chain explicitly constructs them.
The beauty (and usefulness) of $\partial$ was then shown with a proof of the generalized Stokes' theorem.
Which in turn we used in conjunction with generalized partitions to transform an otherwise difficult to compute integral.



Finally, we showed a novel formulation of Petri net graphs. 
Instead of considering transitions as a special type of node, we represented transitions along with corresponding arc weights as a single hybrid set. 
Conditions for liveness and coverability were also discussed.
Relaxing the condition of non-negative markings gives way to \emph{lending Petri nets} \cite{bartolettilending} \cite{bartoletti2013} for which hybrid sets were even better suited. 
Unfortunately, I just discovered the Bartoletti papers and have not had a chance to read more than the abstract.

